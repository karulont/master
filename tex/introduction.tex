\section{Introduction}
\justify
It does not take a scientific paper to see that the number of cars on our
streets has increased. This creates a need for more parking places. It is
common for shopping centers and other public places with lots of guests to have
large parking lots or even parking houses. However, finding a parking space
still is a frustrating experience for the drivers. Some fancy places have valet
parking. Valet parking is rare in Estonia, but it is quite common in North
America. Valet parking means that the driver hand the car over to a valet and a
valet will park and retrieve their car when requested.

One way to further optimize the parking problem - save time and require less 
parking space - is to use automated valet parking.
Instead of a human valet, there is a parking robot that
carries your car to a free parking space. In the system explored in this
thesis, drivers will stop their cars on a special platform. A parking robot can
freely move under those platforms and also can lift the platform up from the
ground and carry it with the car. For a driver the experience should be very 
similar to regular valet parking: driver stops near the door and does not need to worry about parking,
after visiting the venue the driver can notify the system will bring the
car back to them.
\subsection{The problem}
In an automated valet parking system, the software dictates which parking robot 
will service the incoming parking or car retrieval request, it will also need 
to do path-planning for the robots. 
Regular car parks have passageways, so that no car is parked in a way that would block
other cars from entering or exiting. In such automated system we can park cars
more densly, because blocking cars can be lifted and moved out of the way.
It is important that the time taken to retrieve cars is minimized, otherwise
drivers time is wasted waiting. In a car park without passageways, the parking 
robots have to be cooperative and cannot do path planning as independent agents 
like in~\cite{wurman2008coordinating}. There are car parks with different layouts, 
some have large areas connected by narrow corridors, which would break algorithms 
that work on a rectangular grid. Algorithms specially designed for a park layout 
or even car parks built to match the specifics of an algorithm are operational, 
but they need lot of work and generally more expensive. We want to test if a 
general purpose tool like integer programming can be used to optimize automated 
valet parking operations.

There are a lot of different optimization problems in a automated parking system
described. We chose to tackle a sub-problem: given a initial configuration of
robots and cars in the system and a terminal configuration, find the best
sequence of actions for robots to transform the initial configuration into the
terminal configuration. Note that best in this case means mostly shortest time,
but we might also want to give some optimization weights for different actions because 
they waste energy and moving parts wear.

The goal of this thesis was to model the problem as an integer programming
model optimization problem. Unlike heuristic algorithms, integer programming
can give us a strong guarantee that the solution found is in fact optimal.

\subsection{Physical parameters of the problem}
The parking robot can move either forwards, backwards or sideways. The length of
the parking space is 6 meters, and it is 3 meters wide. The robots have smaller
dimensions. Every action takes some time. Lifting a car takes 9 seconds,
lowering it will take 3 seconds. The speed of the robot is 2 meters per second,
when not carrying a car, and 1 meter per second, when loaded. Robots needs time
to accelerate and de-accelerate. Changing the direction takes 3 seconds.

The previous paragraph describes the estimated times which were to be used as
basis for building the model. In the reality they might be different, but this
is an idealized model, because the actual robots are not yet built after all.

It is perfectly valid to interpret the time required for stopping and reversing as acceleration. 
It is said that changing direction takes 3 seconds. The interpretation we used, is
that it takes 1.5 seconds to stop a robot moving at it's maximum speed. And
another 1.5 seconds to accelerate it to maximum speed in another direction.
Maximum speed and acceleration depends on whether the robot is carrying a car or not.
Aaccording to Newton's Second Law, acceleration of robot can be greater when the mass is lower. Meaning that, in this case, a heavy loaded robot can
accelerate at half the rate as unladen one. It might help to
see~\autoref{fig:changedir}.

\begin{figure}[h]
    \begin{center}
        \begin{tikzpicture}[show background rectangle]
\draw[thick,->] (0,0) -- (0,3) node[above] {Speed ($\frac{m}{s}$)};
\draw[thick,->] (0,0) -- (5,0) node[below right] {Time ($s$)};
\draw[very thin, gray] (0,0) grid (4.9,2.9);

\foreach \x in {0,...,4}
    \draw (\x,0.1) -- (\x,-0.1) node [below] {\x};

\foreach \y in {1,...,2}
    \draw (0.1,\y) -- (-0.1,\y) node [left] {\y};

% path
\draw[thick, green] (0,2) -- (1,2) -- (2.5,0) -- (4,2) -- (5,2) node[right] {Empty robot};
\draw[thick, red] (0,1) -- (1,1) -- (2.5,0) -- (4,1) -- (5, 1) node[right] {Loaded robot};
\end{tikzpicture}
\TODO{Maybe improve this figure}

        \caption{This diagram serves as an illustration for the interpretation of
            time it takes to change direction. The direction of movement can change
            instantaneously, when speed is 0. Change of direction is not explicitly shown on
            this graph.}
        \label{fig:changedir}
    \end{center}
\end{figure}

It is assumed that car park is a grid of of parking spaces. We define a set of
directions as $\label{eq:dir}\D = \{\Dn, \De, \Ds, \Dw\}$. A parking place in the $\Dn$ -
$\Ds$ dimension is 6 meters long, and in the $\De$ - $\Dw$ dimension is 3
meters wide. Note that the directions defined here do not have to correspond to
geographical directions.

After those definitions we can say that an empty robot can go to adjacent space
in the $\Dn$ or $\Ds$ direction in 4.5 seconds. Empty robot can go to adjacent
space in the shorter dimension in 3 seconds. A loaded robot needs 7.5 seconds
to reach adjacent parking space in the longer dimension and 4.5 seconds for
shorter dimension. Speed versus time graph of all of those movements is
on~\autoref{fig:moving times}, the area under those lines is equal to the
dimensions of the parking space.
\begin{figure}[h]
    \begin{center}
        \begin{tikzpicture}[show background rectangle]
\draw[thick,->] (0,0) -- (0,3) node[above] {Speed ($\frac{m}{s}$)};
\draw[thick,->] (0,0) -- (8,0) node[below right] {Time ($s$)};
\draw[very thin, gray] (0,0) grid (7.9,2.9);

\foreach \x in {0,...,7}
    \draw (\x,0.1) -- (\x,-0.1) node [below] {\x};

\foreach \y in {1,...,2}
    \draw (0.1,\y) -- (-0.1,\y) node [left] {\y};

% path
\draw[thick, blue] (0,0) -- (1.5,1) -- node[pos=.5, above] {some text} (6,1) -- (7.5,0);
\end{tikzpicture}
\TODO{Improve this figure}

        \caption{This figure shows all the possible movements to an adjacent
        parking space.}
        \label{fig:moving times}
    \end{center}
\end{figure}

When a robot decides to move many spaces in one direction the distance traveled
is equal to the distances covered while accelerating, while de-accelerating and
while moving with full speed. For an empty robot the distance traveled while
accelerating and de-accelerating sums to 3 meters. For each additional parking
place traveled we need 1.5 seconds for $\De$ or $\Dw$ direction. For $\Dn$ or
$\Ds$ direction we first need to add 1.5 seconds for the first space and 3
seconds for each additional parking space. For an loaded robot the distance traveled
while accelerating and de-accelerating sums to 1.5 meters. For the shorter
dimension of the parking space, we need additional 1.5 seconds, to reach a full 3
meters. For additional spaces, we need to add 3 seconds for each additional
space traveled. For the longer dimension, we need additional 4.5 seconds, and
for each additional space we need to add 6 seconds to reach the desired
distance.

As an example, let us consider that a loaded robot wants to move 3
parking spaces in the direction $\Dn$. The total distance needed to cover is
$3\times 6 = 18$ meters. The time it takes is $3 + 4.5 + 2 \times 6 = 19.5s$
seconds. Lets verify that the distance is right: for 3 seconds we are
accelerating and de-accelerating, with top speed 1 $\frac{m}{s}$, the average
speed being 0.5 $\frac{m}{s}$. Other time we are moving with top speed which is 1
$\frac{m}{s}$. The distance traveled will be $3 * 0.5 + 4.5 \times 1 + 2(6
\times 1) = 18$ meters. An additional example is shown
on~\autoref{fig:manyspaces}.
\begin{figure}[h]
    \begin{center}
        \begin{tikzpicture}[show background rectangle]
    \draw[thick,->] (0,0) -- (0,3) node[above] {Speed ($\frac{m}{s}$)};
    \draw[thick,->] (0,0) -- (8,0) node[below right] {Time ($s$)};
    \draw[very thin, gray] (0,0) grid (7.9,2.9);

    \foreach \x in {0,...,7}
    \draw (\x,0.1) -- (\x,-0.1) node [below] {\x};

    \foreach \y in {1,...,2}
    \draw (0.1,\y) -- (-0.1,\y) node [left] {\y};

    % path
    \draw[thick, blue] (0,0) -- (1.5,2) -- node[above] {empty robot} (6,2) --
    (7.5,0);
\end{tikzpicture}


        \caption{This figure shows the movement of an empty robot in $\Dn$ or
            $\Ds$ direction with moving for 2 parking spaces. The total distance
            traveled is 12 meters, which is exactly the length of 2 parking spaces in
        that direction.}
        \label{fig:manyspaces}
    \end{center}
\end{figure}

\begin{table}
    \begin{center}
        \begin{tabular}{| c | c |}
            \hline
            Action & Time (s)\\
            \hline
            Lift car & 9\\
            Drop car & 3\\
            Accelerate to full speed & 1.5\\
            Deacceleration from full speed to stop & 1.5\\
            Loaded robot moving to adjacent space (N or S) & 7.5\\
            Loaded robot moving to adjacent space (E or W) & 4.5\\
            Empty robot moving to adjacent space (N or S) & 3\\
            Empty robot moving to adjacent space (E or W) & 4.5\\
            \hline
        \end{tabular}
        \caption{We list different actions and the time they need to complete.}
        \label{tbl:times}
    \end{center}
\end{table}
\subsection{Formulation as a discrete problem}
\label{sec:discrete problem}
From the~\autoref{tbl:times}, it can be seen that all the times are multiples
of 1.5 seconds. That means we can take 1.5 seconds as a our timestep. All the movements take a discrete number of timesteps and every movement ends up in the exact center of the parking place.

For the car park, we use graph $\label{eq:graph}\mathcal{G} = (\V,\E)$, where every vertex is a
parking space and every edge $(u,v)$ means that there is no obstacle between
parking spaces~$u$ and~$v$. The parking places have degree at most 4,
with the edges corresponding to directions in $\D$.

We let $K$ be the number of different labels we want to give to cars. Cars with
the same label are indistinguishable in our model. Usually there are a lot of
dormant cars and only the cars that have just arrived or requested need to be
given distinctive labels. We do not label robots as they are interchangeable.

The problem is to find a suitable sequence of actions for robots, to
reconfigure the given initial configuration into the terminal configuration.
Both initial and terminal configurations are given by node statuses for every
vertex.

Sometimes a given initial status with given terminal status is unsolvable no
matter how many timesteps are considered. For a very simple example
see~\autoref{fig:unsolvable}.

\begin{figure}[h]
    \begin{center}
        \begin{tikzpicture}[show background rectangle]
% nodes

\foreach \y in {4, 1}
{
    \draw[gray,thick] (0,\y-0.75) -- (5,\y-0.75);
    \draw[fill=gray!30] (0,\y) rectangle (1.5, \y-1.5);
    \draw[fill=gray!30] (2,\y) rectangle (3.5, \y-1.5);
    \draw[fill=gray!30] (4,\y) rectangle (5.5, \y-1.5);
}

\draw[fill=blue] (0.75,3.25) node {0} circle (0.6);
\draw[fill=green] (2.75,3.25) node {R} circle (0.6);
\draw[fill=red] (4.75,3.25) node {1} circle (0.6);

\draw[fill=red] (0.75,0.25) node {1} circle (0.6);
\draw[fill=green] (2.75,0.25) node {R} circle (0.6);
\draw[fill=blue] (4.75,0.25) node {0} circle (0.6);

\node at (2.75, 2) {Initial configuration};
\node at (2.75, -1) {Terminal configuration};

\end{tikzpicture}


        \caption{This figure shows a pair of initial and terminal statuses,
            which cannot be solved. Green robot can move under the cars, but there
            is no way how red car can exchange places with the blue car without
        colliding.}
        \label{fig:unsolvable}
    \end{center}
\end{figure}

\subsubsection{Possible further simplifications}
From theoretical viewpoint, there is a interesting and relevant result
in~\cite{calinescu2008reconfigurations}. For this abstraction, we further
simplify our problem, and also forget about robots and the moving times. We
just have a graph $\mathcal{G}$, initial configuration of cars, and terminal
configuration of cars. Cars can move from one node~$u$ to another node~$v$ in
an instant, the only constraint is that there must be path from $u$ to $v$,
where all intermediate nodes are unoccupied by other cars. The problem is still
to reconfigure the cars, so that terminal configuration is obtained with
minimum number of moves. 

Theorem 1 from~\cite{calinescu2008reconfigurations}, states that for every
connected graph, for every initial and terminal configuration with $n$ cars,
there exists a solution with at most $n$ moves and for trees there is a linear
time algorithm for finding the optimal moves. However, that applies to only
unlabeled cars.

Maybe there are large firms, where everyone can just ride a random company car.
But in general a user expects to receive a specific car. Therefore we still
have to keep cars labeled. 

According to theorem 3 from~\cite{calinescu2008reconfigurations}, with labeled
cars on a graph, the reconfiguration problem is APX-hard. If $P\neq NP$, this
implies that there is no polynomial-time approximation scheme for the problem.

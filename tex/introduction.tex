\section{Introduction}
\subsection{The problem}
\TODO{General bla bla about the parking problem}
\subsection{More specific problem description}
The parking robot can move either forwards, backwards or sideways. The length of
the parking space is 6 meters, and it is 3 meters wide. The robots have smaller
dimensions. Every action takes some time. Lifting a car takes 9 seconds,
lowering it will take 3 seconds. The speed of the robot is 2 meters per second,
when not carrying a car, and 1 meter per second, when loaded. Changing the
direction takes 3 seconds. It needs time to accelerate and de-accelerate. 

The last paragraph describes the information we received. However there is still
some room to interpret the acceleration of the robot. It is said that changing
direction takes 3 seconds. The interpretation we used, is that it takes 1.5
seconds to stop a robot moving at it's maximum speed. And another 1.5 seconds to
accelerate it to maximum speed in another direction. Maximum speed depends on
whether the robot is carrying a car or not, however it makes sense that the
acceleration of robot can be greater, when the mass is lower. Meaning that a
heavy loaded robot can accelerate to speed 1 meter per second and that the empty
robot can accelerate to speed 2 meters per second in the same time. It might
help to see~\autoref{fig:changedir}.

\begin{figure}[h]
    \begin{tikzpicture}[show background rectangle]
\draw[thick,->] (0,0) -- (0,3) node[above] {Speed ($\frac{m}{s}$)};
\draw[thick,->] (0,0) -- (5,0) node[below right] {Time ($s$)};
\draw[very thin, gray] (0,0) grid (4.9,2.9);

\foreach \x in {0,...,4}
    \draw (\x,0.1) -- (\x,-0.1) node [below] {\x};

\foreach \y in {1,...,2}
    \draw (0.1,\y) -- (-0.1,\y) node [left] {\y};

% path
\draw[thick, green] (0,2) -- (1,2) -- (2.5,0) -- (4,2) -- (5,2) node[right] {Empty robot};
\draw[thick, red] (0,1) -- (1,1) -- (2.5,0) -- (4,1) -- (5, 1) node[right] {Loaded robot};
\end{tikzpicture}
\TODO{Maybe improve this figure}

    \caption{This diagram serves as an illustration for the interpretation of
        time it takes to change direction. The direction of movement can change
        instantaneously, when speed is 0. Change of direction is not explicitly shown on
        this graph. }
    \label{fig:changedir}
\end{figure}

\begin{table}
    \begin{tabular}{| c | c |}
        \hline
        Action & Time (s)\\
        \hline
        Lift car & 9\\
        Drop car & 3\\
        Accelerate to full speed & 1.5\\
        Deacceleration from full speed to stop & 1.5\\
        \hline
    \end{tabular}
    \TODO{make this table nice}
    \caption{The table describing the times some action will take}
    \label{tbl:times}
\end{table}
\subsection{Simplified discrete problem}
From the~\autoref{tbl:times}, it can be seen that all the times are multiples
of 1.5 seconds. We still need to verify that the time to move from parking
place to another is indeed a multiple of 1.5 seconds.
In~\autoref{fig:moving-times} all different possible robot movements are shown
with the distance traveled calculated. From the figures it is quite easy to see
that the distance traveled is equal to the dimensions of parking places.

\begin{figure}[h]
    \begin{tikzpicture}[show background rectangle]
\draw[thick,->] (0,0) -- (0,3) node[above] {Speed ($\frac{m}{s}$)};
\draw[thick,->] (0,0) -- (8,0) node[below right] {Time ($s$)};
\draw[very thin, gray] (0,0) grid (7.9,2.9);

\foreach \x in {0,...,7}
    \draw (\x,0.1) -- (\x,-0.1) node [below] {\x};

\foreach \y in {1,...,2}
    \draw (0.1,\y) -- (-0.1,\y) node [left] {\y};

% path
\draw[thick, blue] (0,0) -- (1.5,1) -- node[pos=.5, above] {some text} (6,1) -- (7.5,0);
\end{tikzpicture}
\TODO{Improve this figure}

    \caption{This figure shows different possible movements. }
    \label{fig:moving-times}
\end{figure}

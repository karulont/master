\section{Integer programming model}
\TODO{Write introduction to the model}
Graph of vertices (nodes) of parking places and so forward.
\TODO{sets and iterators}
\TODO{Explanation of movement}
\subsection{Variables}
The variables for the model are all binary, meaning their allowed values are from the set
$\{0,1\}$. There are three groups of variables:
\begin{enumerate}
    \item node status variables,
    \item edge occupied variables,
    \item decision variables.
\end{enumerate}
For every timestep, there is another layer of the variables. Meaning that if
we have $n$ variables for timestep $0$, and we have $t$ timesteps, then in total
there will be $tn$ variables in the model. In the variable indexing, the
timestep is the always the last index.

\subsubsection{Node status variables}
As specified in \autoref{sec:discrete problem}, there are many possible statuses
for parking spots. Node status variables are meant to encode the node status.
There is a variable for every combination of node, status and timestep. To be
more accurate, the group of node status variables is defined as
\begin{align}
    &\forall v \in \V: \forall w \in \W: \forall t \in \T: & \nstat_{v,w,t}
\end{align}

\subsubsection{Edge occupied variables}
There are variables for each directed edge between parking spots. When a robot
moves from vertex $u$ to vertex $v$ at timestep $t$, the edge variable
$\occu_{(u,v),t}$ is 1 to indicate that the edge is occupied. All edge variables
are defined as
\begin{align}
    &\forall e \in \E: \forall t \in \T: & \occu_{e,t}
\end{align}

\subsubsection{Decision variables}
This is the most important group of variables, because they represent the
decisions or actions, that the model should find. The decision variables are
\begin{align}
    &\forall v \in \V: \forall w \in \Wm: \forall d \in \D: \forall t \in \T: &
    \go_{v,w,d,t}\\
    &\forall v \in \V: \forall w \in \Wm: \forall d \in \D: \forall t \in \T: &
    \cont_{v,w,d,t}\\
    &\forall v \in \V: \forall w \in \Wm: \forall d \in \D: \forall t \in \T: &
    \stp_{v,w,d,t}\\
    &\forall v \in \V: \forall w \in \Wl: \forall t \in \T: &
    \lift_{v,w,d,t}\\
    &\forall v \in \V: \forall w \in \Wd: \forall t \in \T: &
    \drop_{v,w,d,t}
\end{align}
Decision variable is set to 1, if and only if that action is taken at the
specified node, with specified node status, and at specified timestep. For
movement variables: $\go_{v,w,d,t}$, $\cont_{v,w,d,t}$ and $\stp_{v,w,d,t}$
also the direction is specified. Decision variables already have
self-describing names, but to clarify, we explain them one-by-one
in~\autoref{tbl:decvars}.

\begin{table}[h]
    \center
    \begin{tabular}{| c | p{\textwidth - 2.6cm} |}
        \hline
        Variable & Meaning\\
        \hline
        $\go_{v,w,d,t}$ & At timestep $t$ on parking place $v$ with node status
        $w$, a robot starts to accelerate in direction $d$.\\ \hline
        $\cont_{v,w,d,t}$ & At timestep $t$ on parking place $v$ with node status
        $w$ a robot that is about to reach the neighbouring node in direction
        $d$, does not slow down and continues moving in direction $d$.\\ \hline
        $\stp_{v,w,d,t}$ & At timestep $t$ on parking place $v$ with node status
        $w$ a robot that is about to reach the neighbouring node in direction
        $d$, starts to de-accelerate to stop in the neighbouring node.\\ \hline
        $\lift_{v,w,t}$ & At timestep $t$ on parking place $v$ with node status
        $w$ a robot starts to lift a car.\\ \hline
        $\drop_{v,w,t}$ & At timestep $t$ on parking place $v$ with node status
        $w$ a robot starts to drop a car.\\
        \hline
    \end{tabular}
    \TODO{make this table nice}
    \caption{The table explaining the meaning of decision variables.}
    \label{tbl:decvars}
\end{table}

\subsection{Constraints}
\subsubsection{Helper functions and notation for constraints}
\label{sec:helpers}
The first helper function used is $\edg : \V \times \D \to \V$. Let $\edg(v,d)$
denote the node that is in the direction $d$ from node $v$.

For directions we use $d + 1$ to denote the next direction from direction $d$ in
clockwise manner. Logically $d - 1$ is used to mark the next direction from $d$
in counterclockwise manner. For an example: $\Dn + 1 = \De$.

Some decisions take a specified amount of time. We want to disallow making such
decisions, which lead to actions, that cannot be completed in the time frame of
the model. For this purpose we want to split $\T$ into two disjoint subsets.
Let $\T_i = \{ t | (t+i) \in \T\}$ and $\Tn_i = \T \setminus \T_i$.
\TODO{Add more meaningful stuff here}
\subsubsection{Simple constraints}
\label{sec:simple}
First constraint is to make sure that at every timestep each node has exactly
one status variable set.
\begin{align}
    &\forall v \in \V: \forall t \in \T: & \sum_{w \in \W} \nstat_{v,w,t} = 1
\end{align}
When a directed edge is used, it's opposite cannot be used at the same time.
\begin{align}
    &\forall (u,v) \in \E: \forall t \in \T: & \occu_{(u,v),t} + \occu_{(v,u),t}
    \leq 1
\end{align}
No more than one moving thing can arrive at the same node at the same time.
\begin{align}
    &\forall v \in \V: \forall t \in \T: & \sum_{u \in \N(v)} \occu_{(u,v),t} \leq 1
\end{align}
We also want to forbid movements in the orthogonal directions. \TODO{Maybe draw
a figure showing the collision we avoid, using these constraints.} Note that there
are vertices such that they do not have a edge between them, or they do
not even have a neighbour in some directions. When that happens, the invalid
$\occu_{e,t}$ variables are not added to the sum.
\begin{equation}
    \begin{split}
        \forall v \in \V: \forall d \in \D: \forall t \in \T: \quad &
        \occu_{(\edg(v,d),v),t} \\ + \occu_{(v,\edg(v,d-1)),t} + &
        \occu_{(v,\edg(v,d+1)),t} \leq 1
    \end{split}
\end{equation}
Also it does not make sense to allow more than 1 decision at the same time, at
the same location.
\begin{equation}
    \begin{split}
        \forall v \in \V: \forall t \in \T: \quad & \sum_{d \in \D, w \in
        \Wm}(\go_{v,w,d,t} + \stp_{v,w,d,t} + \cont_{v,w,d,t}) \\
        + & \sum_{w \in \Wl} \lift_{v,w,t} + \sum_{w \in \Wd} \drop_{v,w,t}
        \leq 1
    \end{split}
\end{equation}

\subsubsection{Lifting and dropping constraints}
These constraints describe the lifting and dropping of cars. In fact they are
quite similar, they indeed are opposite actions. We will start with lifting
constraints. Instead of using $\T$, like we did in~\autoref{sec:simple}, we use
$\T_6$, because lifting takes 6 timesteps. First constraint type is to make sure,
that the node status is correct, when a decision to lift is made.
\begin{align}
    &\forall v \in \V: \forall w \in \Wl: \forall t \in \T_6: &\lift_{v,w,t} -
    \nstat_{v,w,t} \leq 0
\end{align}
While the lifting is in progress, we want the node status be fixed to
$\stat{lft}$.
\begin{align}
    &\forall v \in \V: \forall w \in \Wl: \forall t \in \T_6: \forall i \in
    \{1,\ldots,5\}: &\lift_{v,w,t} -
    \nstat_{v,\stat{lft},t+i} \leq 0
\end{align}
And at the end of lifting the node status should correspond to the lifted
thing. For that we have a simple helper function $f : \Wl \to \Wd$. It is
defined as: $f(\stat{rc}) = \stat{cr}$ and $\forall j: f(\stat{rsc}_j) =
\stat{scr}_j$. In words, the function $f$ determines the node status after
lifting. By using $f$ we can now give the set of constraints.
\begin{align}
    &\forall v \in \V: \forall w \in \Wl: \forall t \in \T_6: &\lift_{v,w,t} -
    \nstat_{v,f(w),t+6} \leq 0
\end{align}
If the lifting cannot be completed, we just make sure that the decision is
never made.
\begin{align}
    &\forall v \in \V: \forall w \in \Wl: \forall t \in \Tn_6: &\lift_{v,w,t} =
    0
\end{align}

For dropping we have almost the same constraints. First make sure that the node
status is correct, when decision is made.
\begin{align}
    &\forall v \in \V: \forall w \in \Wd: \forall t \in \T_2: &\drop_{v,w,t} -
    \nstat_{v,w,t} \leq 0
\end{align}
Dropping take less time than lifting, therefore the constraints for
intermediate node state are simpler than for lifting.
\begin{align}
    &\forall v \in \V: \forall w \in \Wd: \forall t \in \T_2: &\drop_{v,w,t} -
    \nstat_{v,\stat{drp},t+1} \leq 0
\end{align}
To get the correct node status at the end of the drop, we can use the inverse
of the previous helper function $f$.
\begin{align}
    &\forall v \in \V: \forall w \in \Wd: \forall t \in \T_2: &\drop_{v,w,t} -
    \nstat_{v,f^{-1}(w),t+2} \leq 0
\end{align}
As with lifting, if dropping action cannot be completed in the time frame of
the model, disable the decision.
\begin{align}
    &\forall v \in \V: \forall w \in \Wd: \forall t \in \Tn_2: &\drop_{v,w,t} =
    0
\end{align}

\subsubsection{Moving constraints}
Depending on the direction and whether the robot is carrying a car, the
movement takes different number of timesteps. We introduce a new function $g:
\W \times \D \to \mathcal{N}$. Function $g(w,d)$ returns the number of timesteps
the movement of $w$ takes in direction $d$. As with other decisions the first
group of constraints make sure, that node status is correct at the time of the
decision.
\begin{align}
    &\forall u \in \V: \forall w \in \Wm: \forall d \in \D: \forall t \in
    \T_{g(w,d)}:
    & \go_{u,w,d,t} - \nstat_{u,w,t} \leq 0
\end{align}
Next we make sure that $\go_{u,w,d,t}$ is disabled, when movement cannot be
completed.
\begin{align}
    &\forall u \in \V: \forall w \in \Wm: \forall d \in \D: \forall t \in
    \Tn_{g(w,d)}:
    & \go_{u,w,d,t} = 0
\end{align}
When movement is started, it is implied that it should be ended or at continued
at the same node. As a reminder, for the whole duration of before arriving at
the next node, the node status remains the same as when movement started.
\begin{equation}
    \begin{split}
        \forall u \in \V: &\forall w \in \Wm: \forall d \in \D: \forall t \in
        \T_{g(w,d)}:\\
        & \go_{u,w,d,t} - \stp_{u,w,d,t+g(w,d)-1} - \cont_{u,w,d,t+g(w,d)-1} +
        \nstat_{u,w,t} \leq 1
    \end{split}
\end{equation}
Next we make sure that edge occupied variable is 1 for the duration of
movement.
\begin{equation}
    \begin{split}
        \forall u \in \V: \forall w \in \Wm: \forall d \in \D: &\forall t \in
        \T_{g(w,d)}: \forall i \in \{0,\ldots,g(w,d)\}:\\
        & \go_{u,w,d,t} + \nstat_{u,w,t} -\occu_{(u,\edg(u,d),t+i} \leq 1
    \end{split}
\end{equation}
Now we fix the node status for almost the duration of movement.
\begin{equation}
    \begin{split}
        \forall u \in \V: \forall w \in \Wm: \forall d \in \D: &\forall t \in
        \T_{g(w,d)}: \forall i \in \{1,\ldots,g(w,d)-1\}:\\
        & \go_{u,w,d,t} + \nstat_{u,w,t} -\nstat_{(u,w,t+i} \leq 1
    \end{split}
\end{equation}

\subsubsection{Node status constraints}
\subsubsection{Edge occupied constraints}

\noindent\textbf{\large \myTitle}
\vspace*{3ex}
{\flushleft{\textbf{Abstract:}} }
As parking becomes a more and more complex problem with the number of cars and
city density, more complex solutions can be used to rectify it. One of possible
solutions for parking is automated valet parking, where cars are not driven to
parking place by humans but are carried by specially designed robots. Such
solution presents us many possible optimization problems, one of which is
addressed in this work using Integer Programming models, that can by solved
using off-the-shelf solvers. We look at a specific possible implementation of
automated valet parking and succesfully design an Integer Programming model to
solve it. Existing theoretical results have shown that even simplified cases of
the problem can be APX-hard~\cite{calinescu2008reconfigurations}. Using Gurobi,
optimal solution was found for sample cases. The model is compared to another
integer programming model from Algorithms and Theory research group, which
provides comparison and verification for the model performance.

\vspace*{3ex}
{\flushleft{\textbf{Keywords:} integer programming,
\vspace*{3ex}

\noindent\textbf{CERCS:} P170 Computer science, numerical analysis, systems,
control 
\vspace*{3ex}
\newpage
%\selectlanguage{estonian}
\noindent\textbf{\large \myTitlE}
\vspace*{3ex}

{\flushleft{\textbf{Lühikokkuvõte:}} }
\justify
Kuna parkimine on autode hulga suurenemisega ja linnastumise tihenemisega üha
keerulisem ja keerulisem probleem, muutub selle kõrgtehnoloogiline lahendamine
otstarbekaks. Üks pakutud lahendus on automaatparkla, kus autodega ei sõideta
oma parkimiskohta, vaid autod toimetatakse parkimiskohta ja tagasi
spetsiaalsete robotite poolt. Selline kõrgtehnoloogiline lahendus annab meile
palju erinevaid optimiseerimise ülesandeid ja võimalusi, millest ühte
konkreetset käsitleme käesolevas töös kasutades täisarv optimiseerimise
mudeleid, mida saab lahendada juba eksisteerivate analüütiliste lahendajatega.
Käesolevas töös käsitletakse ühte kindlat võimalikku automaatparkla
implementatsiooni ja edukalt tuletatakse täisarv planeerimise mudel selle
lahendamiseks. Varasemad teoreetilised tulemused on näidanud, et isegi
lihtsustatud variandid sellest probleemist saavad olla
APX-keerukusega~\cite{calinescu2008reconfigurations}. Kasutades Gurobi
lahendajat leiti optimaalne lahendus näidis juhtude jaoks. Mudelit võrreldi
teise täisarv planeerimise mudeliga algoritmide ja teooria teadusgruppist, mis
andis kindlust ja võrdlusmaterjali mudeli toimimisele.

\vspace*{3ex}
{\flushleft{\textbf{Keywords:} integer programming,
%\noindent\textbf{Võtmesõnad:} täisarvuline planeerimine
\vspace*{3ex}

\noindent\textbf{CERCS:} P170 Arvutiteadus, arvutusmeetodid, süsteemid,
juhtimine (automaatjuhtimisteooria)
\vspace*{3ex}
%\selectlanguage{english}

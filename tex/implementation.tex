\section{Implementation}
For implementing the model, there were at least two major decisions to be made.
\begin{itemize}
    \item Which optimization software to use?
    \item What programming language to use for generating the model?
\end{itemize}
The author was familiar with a commercial grade optimization solver called
Gurobi\cite{gurobi}. It is advertised as a state-of-the-art mathematical
programming solver and it has a free, full-featured academic license.
Therefore, other integer programming model solvers were not even considered.
\TODO{Maybe should write a bit more about why Gurobi is the best choice?}

Gurobi has bindings for several languages. The ones considered were C++ and
Python. It is well known that Python as a interpreted language has a constant
runtime overhead. However, C++ has lots of unwanted complexity. The author
chose to use Python, because generating the integer programming model would
only take a fraction of the time to actually solve it. And the model solving
is not hindered by the runtime system of Python. Because a there exist lots of
code that still uses version 2.X of Python, it should be explicitly
mentioned that Python 3.X was used in thesis. To be precise, the exact version
was Python 3.5.1.
\subsection{Panda}
Programmers are used to thinking in terms of if clauses. However, in an integer
programming model there are no ifs. At first it was hard to come up with
constraints. You knew which variables are involved and what values are legal,
but writing them as a constraint was still hard.
\subsection{Graphical visualization}
\subsection{Tests written to aid development}
\subsection{Irreducible inconsistent subsystem}
\subsection{Tuning Gurobi parameters}

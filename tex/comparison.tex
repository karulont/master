\section{Comparison}
At the Algorithms and Theory research group in University of Tartu, another
integer programming model was developed with the same problem in mind. The
idea was that we come up with different models, and then can later compare
them. We will denote the model described in this thesis as TH, and the other
model as AT. The AT model used the same simplifications and assumptions described
in~\autoref{sec:discrete problem}. The model has a more variables. With more
variables it is possible use simpler constraints, which contain variables only
from consecutive timesteps. To be more precise, constraints only use the
variables from the current timestep $t$ and from the next timestep $t+1$. To

The objective function for the AT model is a mix between progressive
energy minimization like~\eqref{eq:objective} and minimizing the time when node
statuses different. We estimate that there could be some problems similar to the problem
depicted in~\autoref{apx:movement}. Terminal state is fixed with a large cost,
when it differs in the last timestep. The objective function only considers the
location of cars and not the robots just like the model described
in~\autoref{sec:model}.

\subsection{Testing platform}
The testing was done a machine with Intel(R) Core(TM) i7-4790K CPU.  The
machine has 16GB of operating memory. With hyper-threading the number of
virtual CPUs available was 8. However, the solvers were instructed to use only
one thread, because several instances were run in parallel.

\subsection{Problem instances used for comparison}
The underlying grid layout was taken from a real car park. Initial and terminal
configurations were randomly generated.

\subsection{Results}
Both models were run against several problem instances. Each instance was run 3
times by each model. Number of timesteps for each problem instance was made to
be small, but hopefully enough. The timelimit was passed down to Gurobi
TimeLimit parameter. The time needed to generate the integer programming model
from problem instance is not included in that time. The results show mean Gap
found and the number of cars not in the terminal configuration at~$\tm$. The
results are on~\autoref{tbl:compare}.

\begin{table}[h]
    \begin{center}
        \begin{tabular}{|c|c|c|c|c|c|c|}
            \cline{4-7}
            \multicolumn{3}{l}{\ } & \multicolumn{2}{|c|}{AT} &
            \multicolumn{2}{|c|}{TH}\\
            \hline
            Instance & tlimit & $\tm$ & Gap & C & Gap & C\\
            \hline
            marsi3a-1011 & 1h & 50 & 64.53 & 0 & 96.23 & 2\\
            marsi3a-1111 & 1h & 70 & 97.38 & 2 & 95.74 & 3\\
            marsi3a-2111 & 1h & 50 & 92.70 & 1 & 95.83 & 3\\
            marsi3b-1011 & 1h & 50 & 94.15 & 1 & 94.66 & 2\\
            marsi3b-1111 & 1h & 50 & 74.64 & 0 & 92.92 & 1.66\\
            marsi3b-2111 & 1h & 50 & 68.70 & 0 & 95.68 & 2.66\\
            marsi3c-1011 & 1.5h & 100 & 96.45 & 1 & 95.16 & 2\\
            marsi3c-1111 & 1.5h & 120 & 96.01 & 1 & 95.16 & 2\\
            marsi3c-2111 & 1h & 40 & 43.84 & 0 & 91.81 & 2\\
            marsi3d-1011 & 2h & 150 & 95.85 & 1 & 95.41 & 2\\
            marsi3d-1111 & 2h & 150 & 98.45 & 2 & 95.61 & 3\\
            marsi3d-2111 & 2h & 150 & 97.39 & 1 & 95.21 & 3\\
            marsi3e-1011 & 2h & 150 & 98.72 & 2 & 94.95 & 2\\
            marsi3e-1111 & 2h & 150 & 98.18 & 2 & 94.76 & 3\\
            marsi3e-2111 & 2h & 150 & 97.02 & 1 & 95.17 & 3\\
            \hline
        \end{tabular}
        \caption{Performance measurements of two models. The C column displays
            the number of car that were not delivered to their destinations
        by~$\tm$.}
        \label{tbl:compare}
    \end{center}
\end{table}

\subsection{Interpretation of results}
It seems that AT model finds feasible solutions a lot more than TH model.
However the Gap between incumbent solution and best bound seems to decrease
better with TH model. One observation was that AT model's 3 runs had very
similar bounds. For TH model, the 3 runs of the same problem instance had more
variations.

We also look at a few lines from the Gurobi log for both models.
We start with marsi3c-2111 instance. Gurobi writes the following lines, when
solving AT model.
\begin{verbatim}
Optimize a model with 248610 rows, 70110 columns and 765808 nonzeros
Presolved: 47339 rows, 16854 columns, 199604 nonzeros
Explored 3255 nodes (1267231 simplex iterations) in 3600.01 seconds
\end{verbatim}
The same lines for TH model are:
\begin{verbatim}
Optimize a model with 711882 rows, 64944 columns and 3479938 nonzeros
Presolved: 130560 rows, 38377 columns, 718910 nonzeros
Explored 588 nodes (984243 simplex iterations) in 3600.10 seconds
\end{verbatim}
We can see, that AT model has more columns, but after pre-solving AT model has
less than half of what is left in the TH model. We can also see that in AT
model 3255 MIP nodes were explored, while the TH model only 588 were explored.
From here we can conclude that AT model with easier constraints is better
handled by Gurobi pre-solver.

The next problem instance is marsi3e-1111. For AT model Gurobi generated the
following lines:
\begin{verbatim}
Optimize a model with 1505623 rows, 406492 columns and 4599392 nonzeros
Presolved: 539301 rows, 191201 columns, 2574442 nonzeros
Explored 282 nodes (322789 simplex iterations) in 7200.32 seconds
\end{verbatim}
And the corresponding lines for TH are:
\begin{verbatim}
Optimize a model with 4351099 rows, 376141 columns and 21428351 nonzeros
Presolved: 1042718 rows, 293077 columns, 5719680 nonzeros
Explored 569 nodes (741744 simplex iterations) in 7207.95 seconds
\end{verbatim}
Here we see similar size comparisons between initial model sizes and model
sizes after pre-solve. However, in this case it turn out that Gurobi has
managed to explore more nodes in the TH model, than in AT model. That might
explain, why TH model has a better Gap in this problem instance. 
